Using a Markov chain, we have been able to efficiently sample 2D CDT universes with a fixed volume. On these universes we were able to measure the length profile $\ell(t)$. With this length profile we computed the standard deviation $\sigma_\ell$ and the autocovariance $\rho_\ell(t)$. Using Monte Carlo methods, these were compared to a continuum theory for various system sizes at constant $T/L$ ratio.

Both $\sigma_\ell$ and $\rho_\ell(t)$ seemed to fit the theoretical prediction quite well, though not perfectly. There are a couple of possible sources of errors. Perhaps most importantly, there are still significant statistical errors, mainly visible in Fig. \ref{fig:cov_collapsed}. Furthermore, the theoretical model we used is only valid in the limit $T/L \to \infty$. It is of course not possible to reach this limit in practice. One possible future approach could be to further develop the theory to include higher order corrections. Another option is to investigate how the simulation results scale for different $T/L$ ratios; so far we have only looked at the case $T/L = 20$. However, both of these approaches will likely first require a decrease in statistical errors.

The model we have used is quite efficient, and we have been able to extract some relevant data from it. However, we already have some suggestions that could (somewhat) improve the efficiency of the algorithm. Internally labelling the order four vertices differently would require no updates to the order four list when performing a shard move, and would thus allow for a decrease in computation time. It would also be interesting to investigate whether there is a significant difference in computation time between the two moves. If this is the case, it might be worth it to tune the move rate accordingly, since the correlation time seems to be relatively insensitive to small changes in the move rate. Besides these remarks on implementation, there is still a lot more information that can be extracted from this model; some suggestions have already been discussed in section \ref{sec:observables}. Furthermore, this model can be used as a stepping stone to investigating higher dimensional models, though this is certainly far from trivial.
