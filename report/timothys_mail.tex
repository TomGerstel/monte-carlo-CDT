\paragraph{Timothy's formula's}
Uit hoofdstuk 4.1 van arXiv:1203.3591 kun je afleiden dat in de continue
limiet met een tijdsinterval T en kosmologische constante $\Lambda$ de
covariantie van de ruimtelijke lengtes $L(t)$ in de limiet $T$ naar oneindig gegeven wordt door
\begin{equation}
    \text{Cov}\big(\ell(0), \ell(t)\big) = \Exp{\qty(\ell(0) - L)\qty(\ell(t) - L)}
    = \frac{1}{2\Lambda} e^{- 2\abs{t}\sqrt{\Lambda}}
\end{equation}
terwijl de verwachtingswaarde gelijk is aan
\begin{equation}
    \Exp{\ell(t)} = \frac{1}{\sqrt{\Lambda}} = L
\end{equation}
In deze limiet van grote $T$ is het de verwachting dat het niet veel
uitmaakt of het totale volume vastgezet wordt of dat er een
kosmologische constante genomen wordt (wat is de juiste relatie tussen
volume en $\Lambda$ in dit geval?).

Kun je deze covariantie zien in de data (tenminste voor $\abs{t} \ll T$)? En is
de constante in de $e$-macht daar ook gerelateerd aan $\Exp{L(t)}$?