One could imagine trying to sample triangulations according to the distribution \eqref{eq:full_dist}. However, there is a practical issue with this. It turns out that, due to the structure of $\Omega(n)$, there is a critical value for $\lambda$, namely $\lambda_c = \ln 2$. For $\lambda > \lambda_c$, a typical triangulation sampled from \eqref{eq:full_dist} will contain only a handful of triangles. However, when $\lambda < \lambda_c$, the number of triangles in a typical triangulation will diverge. In the context of Markov Chain Monte Carlo (MCMC), this means that either the triangulations are too small to extract useful data, or an equilibrium is never reached because the triangulation keeps on growing forever.

One solution to this is to keep the total number of triangles fixed during the simulation. Another advantage of this approach is that the desired distribution becomes uniform, since the weight of each triangulation depends only on its volume (number of triangles). This does then require update rules that keep the number of triangles fixed.